\section{把持原理}
\subsection{摩擦力による把持}
本グリッパは式\ref{masatu}に示す摩擦力が把持力となり重力やその他外力とつりあうことで把持を実現する.本グリッパの指は固い円筒状の指の表面が柔軟物が取り付けてある.把持時指を対象物に接触させ力を加えていくと表面の柔軟物は対象物の形状に倣い変形する.このとき指と把持対象物の接触面積が大きくなり摩擦力が増加する.
\begin{equation}
    \label{masatu}
    F = μN
\end{equation}
\subsection{早戻り機構}
早戻り機構\cite{hayamodori}の基本的なメカニズムをレバースライダを例とし\refig{hayamodori}に示す.クランクアームA-QがQを中心に回転すると,レバーがPを中心として往復運動する.クランクアームが一定速度で回転すると点Pから離れる外側を通るときのA-B-Cの角度が240°で,内側のC-D-Aを通るときの角度が120°となりクランクアームが内側の軌道を通る時の時間は外側の軌道のものよりも短くなるので早戻り機構と呼ばれている.\par
早戻り機構を用いたグリッパは内円盤がサーボモータによって回転すると内側のボルトを軸に回転し三本の指が中央に向かって閉じる機構となっている.三本の指の動きがカメラのシャッターのように見えることからシャッターハンドと呼ばれる.指の外側の穴は小判形になっており早戻り機構の高速に動作する範囲のみを抽出できるので高速な開閉を可能にしている.シャッターハンドの開閉の様子を\refig{shutter_hand}に示す.

\begin{figure}[h]
 \begin{center}
  \includegraphics[scale=0.35]{../figure/lever.eps}
 \caption{レバースライダの例}
  \label{fig::hayamodori}
 \end{center}
\end{figure}


\begin{figure}[h]
 \begin{center}
  \includegraphics[scale=0.55]{../figure/shutterhand.eps}
 \caption{シャッターハンド}
  \label{fig::shutter_hand}
 \end{center}
\end{figure}
\newpage

	
