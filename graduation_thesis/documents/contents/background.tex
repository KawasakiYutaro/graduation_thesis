\section{序論}
\label{sec:序論}
産業用ロボットは多様な作業を遂行に対応するためにエンドエフェクタの交換がなされている.エンドエフェクタの一種にグリッパがあり対象物の把持や搬送,組み付けに利用されている.近年,グリッパの交換の省略が期待される汎用性の高いグリッパとしてユニバーサルグリッパの研究開発が行われている.\cite{end}.グリッパの交換の省略は作業環境における最適なグリッパの選定や複雑な把持計画の省略につながり作業の効率化につながる.ユニバーサルグリッパには把持面に柔軟性をもたせたグリッパの開発が進んでおり,このようなロボットに柔軟性を取り入れたソフトロボティクス\cite{soft}という学術分野が近年注目されている.\par
柔軟性のあるグリッパの例にジャミンググリッパ\cite{jam}がある.このグリッパは柔軟な半球状形袋に粒体が封入してある.このグリッパの把持方法は柔軟な状態で把持対象物に押し付け包み込んだ後エアーコンプレッサで袋内の内圧を下げることによって把持部が固化するジャミング現象を利用する.しかし,柔軟膜の耐久性や周囲の気圧などの問題があり長時間の利用が制限される.\par
半球状の柔軟膜の中にMR流体を封入したMR流体グリッパ\cite{MR}がある.
このグリッパはMR流体に磁界を印加することで粘性が変化することを応用し把持に用いる.これらのような柔軟な把持は把持対象物の形状,姿勢に左右にされない強みがある.\par
またグリッパの把持動作の高速化はタクトタイムの短縮になり,産業用ロボットの作業効率化,生産性向上につながると考えられる.指の開閉の高速化により把持動作の高速化を可能としたグリッパに早戻り機構を用いたグリッパがある\cite{sh_hand}.\par
本研究では早戻り機構を用いたグリッパに焦点を当てた.早戻り機構を有するグリッパの把持部に柔軟性をもたせることで高速な開閉が可能なグリッパに把持対象物の形状や姿勢によらない汎用性を付加可能か検証する.

