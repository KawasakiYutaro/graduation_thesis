

\documentclass[a4paper,12pt]{jarticle}

\input{include/macro.tex}
\input{include/preamble.tex}

% 表紙
\title{卒業論文\\
早戻り機構を有するロボットハンドへの柔軟指の導入\\
% {\large No title}
}
\author{\vspace{90mm}\\
指導教員:\ 西田 \hspace{0mm} 健 准教授\\
九州工業大学\ \hspace{0mm} 工学部\\
機械知能工学科\ \hspace{0mm} 知能制御工学コース \\
\vspace{0mm}\\
学籍番号:\ 15104026\\
提出者氏名:\ 川崎 \hspace{0mm} 雄太朗\\\vspace{5mm}\\ }
\date{平成31年\ 2月\ 13日}

\begin{document}

\titlepage
\maketitle
\thispagestyle{empty}
\newpage

% 目次
\thispagestyle{empty}
\input{contents/abstruct.tex}
\newpage
\tableofcontents
% 最初のやつ

\newpage
\input{contents/background.tex}
\newpage
%\input{contents/structure.tex}
\newpage
\input{contents/principle.tex}
\newpage
\input{contents/experiments.tex}
\newpage
\input{contents/discussion.tex}
\newpage
\input{contents/conclutions.tex}
\input{contents/acknowledgement.tex}



\begin{thebibliography}{99}
\addcontentsline{toc}{section}{参考文献}

\bibitem{end} Tetsuyou Watanabe, Kimitoshi Yamazaki, Yasuyoshi Yokoko-hji, “Survey of robotic manipulation studies intending practi-cal applications in real environments -object recognition, softrobot hand, and challenge program and benchmarking-” , Advanced Robotics , Vol. 31, Iss. 19–20, pp. 1114-1132, 2017.
\bibitem{soft} 中村太郎,”生物・生体の機能を規範としたソフトロボティクス”,システム/制御/情報, 61巻, 7号, pp.265-270, 2017.

\bibitem{jam}J. R. Amend Jr, E. Brown, N. Rodenberg, H. M. Jaeger, H. Lipson: “A Positive Pressure
Universal Gripper Based on the Jamming of Granular Material,” IEEE trans. on Robotics,
Vol.28, pp.341–350, 2012.

\bibitem{MR}T . Nishida , Y . Okatani , K . Tadakuma ,
“ Development of universal robot gripper using
MR α fluid, ” Int . Journal of Humanoid Robotics , vol.13 ,No.4 ,p13,2016.

\bibitem{sh_hand} 天呑将成,鈴木陽介,辻徳生,渡辺哲陽,”はや戻り機構を用いた高速グリッパの開発", SICE, 2018.

\bibitem{hayamodori} 熊谷英樹,”必携「からくり設計」メカニズム定石集”,日刊工業新聞社,p140,2017.

\bibitem{gel} 柴山充弘,”ゲルの物理と化学の新展開”,日本物理学会誌 Vol 72,No.4,2017, pp.226-227, 2017.








  %\bibitem{Muhammad} Raúl Mur-Artal, J. M. M. Montiel and Juan D. Tardós. ORB-SLAM: A Versatile and Accurate Monocular SLAM System. IEEE Transactions on Robotics, vol. 31, no. 5, pp. 1147-1163, 2015.
  

%\bibitem{OSRF} ”Open Robotics", https://www.osrfoundation.org/.
%\bibitem{ROS_book}"西田健,森田賢,岡田浩之,原祥堯,山崎公俊,田向権,垣内洋平,大川一也,斎藤功,田中良道,有田裕太,石田祐太郎","実用ロボット開発のためのROSプログラミング.”,森北出版株式会社,pp.1-2, 2018.  
%\bibitem{ROS_book_PCL}"西田健,森田賢,岡田浩之,原祥堯,山崎公俊,田向権,垣内洋平,大川一也,斎藤功,田中良道,有田裕太,石田祐太郎","実用ロボット開発のためのROSプログラミング.”,森北出版株式会社,pp.5, 2018.  


\end{thebibliography}

% 付録の始まり
\appendix
%\def\thesection{付録\Alph{section}}
%\def\thesubsection{\Alph{section}\arabic{subsection}}

%\makeatletter
%\renewcommand{\theequation}{\Alph{section}.\arabic{equation}}
%\@addtoreset{equation}{section}
%\makeatother
% \setcounter{page}{1}
% ODEの説明
%\section{ }
%\label{sec:}


% % 図の挿入

% \begin{figure}[b]
%  \begin{center}
%   \includegraphics[scale=0.9]{../figure/circuit.eps}
%   \caption{Uncontrolled converter}
%   \label{circuit}
%  \end{center}
% \end{figure}


% % 表の挿入

% \begin{table}[htb]
%   \begin{center}
%     \caption{各素子のパラメータ}
%     \begin{tabular}{c|c|c} \hline
%       定数名[単位] & 記号 & 値 \\ \hline \hline
%       周波数[Hz] & $f_U,f_V,f_W$ & 120 \\ \hline
%                      & $\phi_U$ & $\frac{2\pi}{3}$ \\
%       初期位相角[rad] & $\phi_V$ & $\frac{4\pi}{3}$ \\
%                      & $\phi_W$ & $2\pi$ \\ \hline
%       抵抗[$\Omega$] & $R$ & 10 \\ \hline
%     \end{tabular}
%     \label{param}
%   \end{center}
% \end{table}


% % 図の挿入

% \begin{figure}[tb]
%  \centering
%  \vspace{0.5cm}
%  \includegraphics[scale=0.85]{../figure/waves.eps}\\
%  \hspace{0.0cm}
%  % 入力と出力\\
%  % \\
%  % \vspace{1.2cm}
%  % \includegraphics[scale=0.825]{../figure/output.eps}\\
%  % (b) 出力の電位\\
%  % \\
%  \caption{シミュレーションにより得られた各電源電圧(上)と出力電位(下)の波形}
%  \label{wave}
% \end{figure}

% \newpage

% % 図を並べて挿入

% \begin{figure}[tb]
%  \centering
%  \subfloat[区間1における回路動作]{\includegraphics[scale=0.5]{../figure/kukan_1.eps}}
%  \hspace{1.5cm}
%  \subfloat[区間2における回路動作]{\includegraphics[scale=0.5]{../figure/kukan_2.eps}}
% \\
%  \vspace{0.5cm}
%  \subfloat[区間3における回路動作]{\includegraphics[scale=0.5]{../figure/kukan_3.eps}}
%  \hspace{1.5cm}
%  \subfloat[区間4における回路動作]{\includegraphics[scale=0.5]{../figure/kukan_4.eps}}
% \\
%  \vspace{0.5cm}
%  \subfloat[区間5における回路動作]{\includegraphics[scale=0.5]{../figure/kukan_5.eps}}
%  \hspace{1.5cm}
%  \subfloat[区間6における回路動作]{\includegraphics[scale=0.5]{../figure/kukan_6.eps}}
% \\
%  \caption{各区間での回路動作の様子}
%  \label{circuit_kaku}
% \end{figure}

% % 文中へのラベリング
% {\bf Fig. }\ref{circuit_kaku}に示す〜



\end{document}
